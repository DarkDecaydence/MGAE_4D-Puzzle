\section{Introduction}
 This paper is a technical account, describing the design and implementation of the game 'Redshift', developed using the Unity game engine. The paper is split into six sections, the first of which is this introduction. In the following section, we will present the initial concept idea and the vision behind the game, followed by a section dedicated to the chronological account of the development process. The fourth section is concerned with discussing the implementation choices and alternative solutions, evaluating the final implementation. The fifth section concludes upon the work, and leaves the final section dedicated to bibliography and appendix.
 
\subsection{Scope and Context}
The development of 'Redshift' was the subject of two exam projects at the IT-University of Copenhagen, as part of the courses 'Game Design' and 'Game Engines' completed during the fall semester of 2015.\\

The scope of these two projects are very different, as Game Design is focused on the design of mechanics and aesthetics of games, whereas Game Engines is focused on working with game engines specifically, i.e. the implementation of features in game engines, and the related software architecture.\\

This paper is part of the Game Engines course, thus the subjects pertaining to the game's design and its mechanics is not within the scope of this paper. However, the focus will be on the implementation of said mechanics.\\

The projects done for Game Design are usually technically simple, and often done with 2D-graphics. Merging the Game Design and Game Engines projects allowed for more technically challenging features and mechanics.\\

The team on the Game Design project included two additional members. These members had only very limited technical experience and was assigned the roles of designers. Their tasks were mainly asset creation (i.e. sounds, texture, models) and level design. They did not work on the scripts, and did only limited work in the Unity editor. In the final product only one of the levels and the rotating cube special effect at the end of each level was made by one of the designers.



