\section{Project Description}
The goal of this project was create a puzzle game using the Unity 3d-engine. This game should incorporate a fourth spatial dimension, making it a 4d-puzzle game, which require a fourth dimension being implemented in the Unity 3d-engine.\\

The fourth dimension is achieved by building several 3d-environments which inhabits the same 3d-space. Traversing the fourth axis, W, is to traverse the different environments. Thus the W-axis can only be traversed in incremental steps where each step moves the player from one 3d-environment to another. 
\subsection{Design and Features}
The design and desired features of the game can be divided into two groups: 4d-mechanics and 3d-mechanics. This distinction separates the design into features that are unique to the 4d-design of the game, and the features which still functions with the absence of a fourth spatial dimension.
\subsubsection{3d-Mechanics}
These are the desired features which are not dependent on the existence of a fourth spatial dimension. 
The following list aims to define and clarify each feature and their dependence on other parts of the design.
\begin{itemize}
	\item 3d-FPS movement: A player character controlled from a first person perspective using the keyboard and mouse. Allowing the player to traverse, perceive and interact with the game world. Unity offers several built-in prefabs consisting of game objects, cameras and scripts which fulfills the need of a character controller.
	\item Picking up and moving objects: The player must be able to pick up certain game objects. When an object is picked up, it must be held in front of the player character until dropped, allowing the player to move it from one place in the game world to another. Unity Lessons\cite{unityLessons} offered scripts for a basic implementation that allows for a single objects to be picked up and moved. However it has to be expanded to allow for more complex objects to be used, eg. several objects linked as one. It also had to be tweaked since the original scripts made a picked up object kinematic allowing it go through walls.  
	\item Moving objects: Certain objects like doors must be able to move or rotate. Using a single generic script allows for both moving, rotating and scaling objects. 
	\item Interactible objects: The player must be able to interact with certain game objects. These objects can be used like buttons with various effects like opening a door or calling an elevator. The interactible objects must also have the ability to be locked and requiring a certain key to be present. The interactible objects must be implemented in such a way that it can be used in combination with other parts of the design eg. moving an object by pressing a button. Implementing both a generic script and an interface allows for easy interaction between objects and for special cases offers a template usable with the rest of the design.
	\item Display/Monitor: In-game displays, like a computer monitor or TV-screen, is a requirement of more advanced puzzles relying on multiple buttons and combinations, eg. the player has to operate a machine. Using the Unity's cameras and render textures this can be achieved in the Unity editor alone. 
	\item UI Mini-map: In order to help the player navigate the game world a mini-map is required, offering a top down view of the environment. Using cameras and render textures like the in-game displays this is achieved inside the Unity editor. Suspending a downwards facing camera over the player and linking it to a render texture in the UI.  
	
\end{itemize}
\subsubsection{4d-Mechanics}
\begin{itemize}
	\item Collisions: Objects in the same 3d-space but on different points on the fourth axis must not collide. Meaning that even though two objects are intersecting in the 3d-environment they should only collide if they also intersect on fourth axis. Using Unity's 3d-collision detection and expanding it by using Unity's layers system as collision layers an incremental 4th axis can be achieved. This requires objects to change layers according to their W position in order to for the collisions to work correctly. Using scripts to add a W coordinate and handle the layering offers a fairly simple implementation. However many objects require this mechanism thus it is important that its optimized both in terms of CPU time and memory usage and also for use in level design in Unity. Therefore a script which can handle multiple nested game objects was also required. 
	\item Moving: In relation to Collisions and interaction the player controller must also change layer according to its W-position. The controller must also offer the ability to change the W position. This requires the mechanism of detecting interactible objects must be filtered according to W position. 
	\item Rendering/Coloring: Objects must be rendered differently according to their W location, otherwise the player would be able to see everything at all times, which would only confuse the player there would be no difference on what walls the player can or can not collide with. Changing colors, fading objects making them transparent and making them completely invisible are all different ways of helping the player to perceive the 4d-nature of the game world. Using scripts to detect the change of the players W position and change the rendering of the game objects facilitates the world changing around the player as they traverse the fourth axis.
\end{itemize}

