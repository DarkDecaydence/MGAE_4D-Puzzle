\section{Project Description}
The goal of this project was to create a puzzle game using the Unity 3D-engine, that incorporates a fourth spatial dimension; a 4D-puzzle game. This requires the implementation of a fourth dimension in the Unity 3D-engine.\\

The fourth dimension is achieved by building several 3D-environments which inhabit the same 3D-space, yet is separated on the fourth axis (W). The W-axis can be traversed in incremental steps where each step moves the player from one 3D-environment to another.
 
\subsection{Design and Features}
The design and desired features of the game can be divided into two groups; 4D-mechanics and 3D-mechanics. This distinction separates the design into features that are unique to the game, and those which still function with the absence of a fourth spatial dimension.

\subsubsection{3d-Mechanics}
These are the desired features which are not dependent on the existence of a fourth spatial dimension. The following list aims to define and clarify each feature and their dependence on other parts of the design.
\begin{itemize}
	\item 3D-FPS movement:\\
	A player character controlled from a first person perspective using the keyboard and mouse, allowing the player to traverse, perceive, and interact with the game world. Unity offers several built-in prefabs consisting of game objects, cameras and scripts which fulfils this role.
	\item Picking up and moving objects:\\
	The player must be able to pick up certain game objects. When an object is picked up, it must be held in front of the player character until dropped, allowing the player to move it from one place in the game world to another. Unity Lessons\cite{unityLessons} offered scripts for a basic implementation that allows for a single object to be picked up and moved. However it has to be expanded, allowing for more complex objects to be used, eg. several objects linked as one. It also had to be tweaked since the original scripts made a carried object kinematic, allowing it go through walls.  
	\item Moving objects:\\
	Certain objects like doors must be able to move or rotate. Using a single generic script allows for both moving, rotating, and scaling objects. 
	\item Interactable objects:\\
	The player must be able to interact with certain game objects, such as buttons which can be used with various effects, e.g. opening a door or calling an elevator. The interactable objects must also have the ability to be locked and requiring a certain key to be present. The interactable objects must be implemented in such a way that it can be used in combination with other parts of the design, e.g. moving an object by pressing a button. Implementing both a generic script and an interface allows for easy interaction between objects and offers a template such that special cases are easily implemented with the rest of the design.
	\item Display/Monitor:\\
	In-game displays, like a computer monitor or TV-screen, is a requirement for more advanced puzzles relying on multiple buttons and combinations, eg. the player has to operate a machine. Using the Unity's cameras and render textures this can be achieved in the Unity editor alone. 
	\item UI Mini-map:\\
	In order to help the player navigate the game world, a mini-map is required offering a top down view of the environment. Using cameras and render textures like the in-game displays, this is achieved inside the Unity editor, suspending a downwards facing camera over the player and linking it to a render texture in the UI.
\end{itemize}

\subsubsection{4D-Mechanics}
\begin{itemize}
	\item Collisions:\\
	Objects in the same 3d-space, but on different points along the fourth axis, must not collide, meaning that even though two objects are intersecting in the 3d-environment they should only collide if they also intersect on the fourth dimension. Using Unity's 3D-collision detection and expanding it through the layers system (making collision layers), an incremental 4th axis can be achieved. This requires objects to change layers according to their W position for the collisions to work correctly. Using scripts to add a W coordinate, and handle the layering, creates a fairly simple implementation. However, many objects require this mechanism, thus it is important for it to be optimized in terms of processing time, memory usage, and accessibility in level design. Therefore a script which can handle multiple nested game objects was also required. 
	\item Moving:\\
	In relation to Collisions and interaction, the player controller must also change layer according to its W-position, while granting the player an ability to change their W-position. This requires that the mechanism used for detecting interactable objects is filtered according to W-position.
	\item Rendering/Colouring:\\
	Objects must be rendered differently dependant on their W-position, or the player would be able to see everything at all times, with no obvious way to distinguish what walls the player will collide with. Changing colours and fading objects, making them transparent or completely invisible, are different ways of helping the player perceive the game's 4D nature. Using scripts to detect the change of the players W-position, and change the rendering of game objects accordingly, facilitates the world changing around the player when they traverse the fourth axis.
\end{itemize}